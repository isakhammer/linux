\documentclass{article}
\usepackage[utf8]{inputenc}

\title{Linear Methods Notes}
\author{isakhammer }
\date{July 2020}

\usepackage{natbib}
\usepackage{graphicx}
\usepackage{amsmath}
\usepackage{amsthm}
\usepackage{amsfonts}
\usepackage{mathtools}
\usepackage{enumerate} 

\theoremstyle{definition}
\newtheorem{definition}{Definition}[section]
\newtheorem{proposition}{Proposition}[section]


%\newcommand{\norm}[1]{\left\lVert#1\right\rVert}


\DeclarePairedDelimiter\abs{\lvert}{\rvert}%
\DeclarePairedDelimiter\norm{\lVert}{\rVert}%

% Swap the definition of \abs* and \norm*, so that \abs
% and \norm resizes the size of the brackets, and the 
% starred version does not.
\makeatletter
\let\oldabs\abs
\def\abs{\@ifstar{\oldabs}{\oldabs*}}
%
\let\oldnorm\norm
\def\norm{\@ifstar{\oldnorm}{\oldnorm*}}
\makeatother


\begin{figure}[ht]
    \centering
    \incfig{something}
    \caption{something}
    \label{fig:something}
\end{figure}


\theoremstyle{remark}
\newtheorem*{remark}{Remark}


\begin{document}

\maketitle

\section{Sequences in metric spaces and normed spaces}

\begin{definition} {Sequences in metric space}
Let \( (X,d) \) be a metric space. A sequence \( (x_n)_{n \in \mathbb{N}} \) in \( X \) is said to \textbf{converge to} \( x \in X \) if for every \(\epsilon > 0 \) once can find $N = N(\epsilon) \in \mathbb{N}$ such that

 \begin{equation}
   C = d(x_n,x) < \epsilon
\end{equation}

whenever $n \geq N$. The element $x$ is called the \textbf{limit} of the sequence $(x_n)_{n \in \mathbb{N}}$.

In particular, if $(X, \norm{.})$ is a normed space, then $(x_n)_{n \in \mathbb{N}}$ converges to $x \in X$ if for every $\epsilon > 0$ once can find $N = N(\epsilon) \in N $ such that 

\begin{equation}
    \norm{x - x_n} < \epsilon
\end{equation}

whenever $n \geq N$.
\end{definition}


\begin{proposition}{Properties}

Supposed that the sequence $(x_n)_{n \in \mathbf{N}} $ in the normed space $(X, \norm{.})$ is convergent. 
\begin{enumerate}[(i)]
\item The limit $x \ in X$ of the sequence $(x_n)_{n \in \mathbf{N}}$ is unique.
\item The norm of $x_n$ converges to the norm of $x$,
    \begin{equation}
        \abs{\norm{x} - \norm{x_n}  } \rightarrow  0
    \end{equation}

\end{enumerate}


\end{proposition}


\begin{proof}

\begin{enumerate} [(i)]
    \item
        Suppose there exist limits $x,y \in X$ of $(x_n)_{n \in \mathbb{N}}$. Then for any $\epsilon  > 0$ there exist $N_1, N_2 \in \mathbb{N}$ such that $\norm{x_n - n} \leq \epsilon /2$ for all $n \geq N_1$, and $\norm{x_n -y } \leq  \epsilon / 2 $ for all $n \geq N_2$. Hence for all $n \geq \max \{N_1, N_2 \}$
        
        \begin{equation*}
            \norm{x-y} = \norm{x- x_n + x_n-y} \leq \norm{x- x_n} + \norm{x_n-y} \leq \epsilon
        \end{equation*}
    
    \item
        By the reversed triangle inequality we have that 
        \begin{equation*}
            \abs{\norm{x_n} -  \norm{x}}  \leq \norm{x_n -x},
        \end{equation*}
        
        and by assumption $\norm{x_n - x } \rightarrow 0$
\end{enumerate}

    
\end{proof}









\section{Open and Closed Sets}

\begin{definition} {Open Set}

A set \(U \) in a metric space \((M,d) \) is called an \textbf{open set} if U contains a neighborhood of each its points. In other words, U is an open set if, given \( x \in U \), there is some \( \epsilon > 0 \) such that \( B_{\epsilon} \subset U \). 

\end{definition}


\begin{proposition}

For any \( x \in M \) and any \( \epsilon > 0 \) the open ball \( B_{\epsilon}(x) \) is an open set. 

\end{proposition}

\begin{proof}

Let \( y \in B_{\epsilon}(x) \). Then \( d(x,y) < \epsilon \) and hence \( \delta = \epsilon -  d(x,y)  > 0\). We will show that \( B_{\delta}(y) \subset B_{\epsilon}(x)\). Indeed, if \( d(y,z) < \epsilon \), then, by the triangle equality, \( d(x,y)  \leq  d(x,y) + d(y,z) +  \epsilon = d(x,y) +  \epsilon - d(x,y) = \epsilon \). 

\end{proof}


\begin{definition} {Closed Set}


\end{definition}





\bibliographystyle{plain}
\bibliography{references}
\end{document}


